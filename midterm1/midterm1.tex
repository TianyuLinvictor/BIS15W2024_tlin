% Options for packages loaded elsewhere
\PassOptionsToPackage{unicode}{hyperref}
\PassOptionsToPackage{hyphens}{url}
%
\documentclass[
]{article}
\usepackage{amsmath,amssymb}
\usepackage{iftex}
\ifPDFTeX
  \usepackage[T1]{fontenc}
  \usepackage[utf8]{inputenc}
  \usepackage{textcomp} % provide euro and other symbols
\else % if luatex or xetex
  \usepackage{unicode-math} % this also loads fontspec
  \defaultfontfeatures{Scale=MatchLowercase}
  \defaultfontfeatures[\rmfamily]{Ligatures=TeX,Scale=1}
\fi
\usepackage{lmodern}
\ifPDFTeX\else
  % xetex/luatex font selection
\fi
% Use upquote if available, for straight quotes in verbatim environments
\IfFileExists{upquote.sty}{\usepackage{upquote}}{}
\IfFileExists{microtype.sty}{% use microtype if available
  \usepackage[]{microtype}
  \UseMicrotypeSet[protrusion]{basicmath} % disable protrusion for tt fonts
}{}
\makeatletter
\@ifundefined{KOMAClassName}{% if non-KOMA class
  \IfFileExists{parskip.sty}{%
    \usepackage{parskip}
  }{% else
    \setlength{\parindent}{0pt}
    \setlength{\parskip}{6pt plus 2pt minus 1pt}}
}{% if KOMA class
  \KOMAoptions{parskip=half}}
\makeatother
\usepackage{xcolor}
\usepackage[margin=1in]{geometry}
\usepackage{color}
\usepackage{fancyvrb}
\newcommand{\VerbBar}{|}
\newcommand{\VERB}{\Verb[commandchars=\\\{\}]}
\DefineVerbatimEnvironment{Highlighting}{Verbatim}{commandchars=\\\{\}}
% Add ',fontsize=\small' for more characters per line
\usepackage{framed}
\definecolor{shadecolor}{RGB}{248,248,248}
\newenvironment{Shaded}{\begin{snugshade}}{\end{snugshade}}
\newcommand{\AlertTok}[1]{\textcolor[rgb]{0.94,0.16,0.16}{#1}}
\newcommand{\AnnotationTok}[1]{\textcolor[rgb]{0.56,0.35,0.01}{\textbf{\textit{#1}}}}
\newcommand{\AttributeTok}[1]{\textcolor[rgb]{0.13,0.29,0.53}{#1}}
\newcommand{\BaseNTok}[1]{\textcolor[rgb]{0.00,0.00,0.81}{#1}}
\newcommand{\BuiltInTok}[1]{#1}
\newcommand{\CharTok}[1]{\textcolor[rgb]{0.31,0.60,0.02}{#1}}
\newcommand{\CommentTok}[1]{\textcolor[rgb]{0.56,0.35,0.01}{\textit{#1}}}
\newcommand{\CommentVarTok}[1]{\textcolor[rgb]{0.56,0.35,0.01}{\textbf{\textit{#1}}}}
\newcommand{\ConstantTok}[1]{\textcolor[rgb]{0.56,0.35,0.01}{#1}}
\newcommand{\ControlFlowTok}[1]{\textcolor[rgb]{0.13,0.29,0.53}{\textbf{#1}}}
\newcommand{\DataTypeTok}[1]{\textcolor[rgb]{0.13,0.29,0.53}{#1}}
\newcommand{\DecValTok}[1]{\textcolor[rgb]{0.00,0.00,0.81}{#1}}
\newcommand{\DocumentationTok}[1]{\textcolor[rgb]{0.56,0.35,0.01}{\textbf{\textit{#1}}}}
\newcommand{\ErrorTok}[1]{\textcolor[rgb]{0.64,0.00,0.00}{\textbf{#1}}}
\newcommand{\ExtensionTok}[1]{#1}
\newcommand{\FloatTok}[1]{\textcolor[rgb]{0.00,0.00,0.81}{#1}}
\newcommand{\FunctionTok}[1]{\textcolor[rgb]{0.13,0.29,0.53}{\textbf{#1}}}
\newcommand{\ImportTok}[1]{#1}
\newcommand{\InformationTok}[1]{\textcolor[rgb]{0.56,0.35,0.01}{\textbf{\textit{#1}}}}
\newcommand{\KeywordTok}[1]{\textcolor[rgb]{0.13,0.29,0.53}{\textbf{#1}}}
\newcommand{\NormalTok}[1]{#1}
\newcommand{\OperatorTok}[1]{\textcolor[rgb]{0.81,0.36,0.00}{\textbf{#1}}}
\newcommand{\OtherTok}[1]{\textcolor[rgb]{0.56,0.35,0.01}{#1}}
\newcommand{\PreprocessorTok}[1]{\textcolor[rgb]{0.56,0.35,0.01}{\textit{#1}}}
\newcommand{\RegionMarkerTok}[1]{#1}
\newcommand{\SpecialCharTok}[1]{\textcolor[rgb]{0.81,0.36,0.00}{\textbf{#1}}}
\newcommand{\SpecialStringTok}[1]{\textcolor[rgb]{0.31,0.60,0.02}{#1}}
\newcommand{\StringTok}[1]{\textcolor[rgb]{0.31,0.60,0.02}{#1}}
\newcommand{\VariableTok}[1]{\textcolor[rgb]{0.00,0.00,0.00}{#1}}
\newcommand{\VerbatimStringTok}[1]{\textcolor[rgb]{0.31,0.60,0.02}{#1}}
\newcommand{\WarningTok}[1]{\textcolor[rgb]{0.56,0.35,0.01}{\textbf{\textit{#1}}}}
\usepackage{graphicx}
\makeatletter
\def\maxwidth{\ifdim\Gin@nat@width>\linewidth\linewidth\else\Gin@nat@width\fi}
\def\maxheight{\ifdim\Gin@nat@height>\textheight\textheight\else\Gin@nat@height\fi}
\makeatother
% Scale images if necessary, so that they will not overflow the page
% margins by default, and it is still possible to overwrite the defaults
% using explicit options in \includegraphics[width, height, ...]{}
\setkeys{Gin}{width=\maxwidth,height=\maxheight,keepaspectratio}
% Set default figure placement to htbp
\makeatletter
\def\fps@figure{htbp}
\makeatother
\setlength{\emergencystretch}{3em} % prevent overfull lines
\providecommand{\tightlist}{%
  \setlength{\itemsep}{0pt}\setlength{\parskip}{0pt}}
\setcounter{secnumdepth}{-\maxdimen} % remove section numbering
\ifLuaTeX
  \usepackage{selnolig}  % disable illegal ligatures
\fi
\IfFileExists{bookmark.sty}{\usepackage{bookmark}}{\usepackage{hyperref}}
\IfFileExists{xurl.sty}{\usepackage{xurl}}{} % add URL line breaks if available
\urlstyle{same}
\hypersetup{
  pdftitle={Midterm 1 W24},
  pdfauthor={Tianyu Lin},
  hidelinks,
  pdfcreator={LaTeX via pandoc}}

\title{Midterm 1 W24}
\author{Tianyu Lin}
\date{2024-02-06}

\begin{document}
\maketitle

\hypertarget{instructions}{%
\subsection{Instructions}\label{instructions}}

Answer the following questions and complete the exercises in RMarkdown.
Please embed all of your code and push your final work to your
repository. Your code must be organized, clean, and run free from
errors. Remember, you must remove the \texttt{\#} for any included code
chunks to run. Be sure to add your name to the author header above.

Your code must knit in order to be considered. If you are stuck and
cannot answer a question, then comment out your code and knit the
document. You may use your notes, labs, and homework to help you
complete this exam. Do not use any other resources- including AI
assistance.

Don't forget to answer any questions that are asked in the prompt!

Be sure to push your completed midterm to your repository. This exam is
worth 30 points.

\hypertarget{background}{%
\subsection{Background}\label{background}}

In the data folder, you will find data related to a study on wolf
mortality collected by the National Park Service. You should start by
reading the \texttt{README\_NPSwolfdata.pdf} file. This will provide an
abstract of the study and an explanation of variables.

The data are from: Cassidy, Kira et al.~(2022). Gray wolf packs and
human-caused wolf mortality.
\href{https://doi.org/10.5061/dryad.mkkwh713f}{Dryad}.

\hypertarget{load-the-libraries}{%
\subsection{Load the libraries}\label{load-the-libraries}}

\begin{Shaded}
\begin{Highlighting}[]
\FunctionTok{library}\NormalTok{(}\StringTok{"tidyverse"}\NormalTok{)}
\FunctionTok{library}\NormalTok{(}\StringTok{"janitor"}\NormalTok{)}
\end{Highlighting}
\end{Shaded}

\hypertarget{load-the-wolves-data}{%
\subsection{Load the wolves data}\label{load-the-wolves-data}}

In these data, the authors used \texttt{NULL} to represent missing
values. I am correcting this for you below and using \texttt{janitor} to
clean the column names.

\begin{Shaded}
\begin{Highlighting}[]
\NormalTok{wolves }\OtherTok{\textless{}{-}} \FunctionTok{read.csv}\NormalTok{(}\StringTok{"data/NPS\_wolfmortalitydata.csv"}\NormalTok{, }\AttributeTok{na =} \FunctionTok{c}\NormalTok{(}\StringTok{"NULL"}\NormalTok{)) }\SpecialCharTok{\%\textgreater{}\%} \FunctionTok{clean\_names}\NormalTok{()}
\end{Highlighting}
\end{Shaded}

\hypertarget{questions}{%
\subsection{Questions}\label{questions}}

Problem 1. (1 point) Let's start with some data exploration. What are
the variable (column) names?

\begin{Shaded}
\begin{Highlighting}[]
\FunctionTok{names}\NormalTok{(wolves)}
\end{Highlighting}
\end{Shaded}

\begin{verbatim}
##  [1] "park"         "biolyr"       "pack"         "packcode"     "packsize_aug"
##  [6] "mort_yn"      "mort_all"     "mort_lead"    "mort_nonlead" "reprody1"    
## [11] "persisty1"
\end{verbatim}

Problem 2. (1 point) Use the function of your choice to summarize the
data and get an idea of its structure.

\begin{Shaded}
\begin{Highlighting}[]
\FunctionTok{summary}\NormalTok{(wolves)}
\end{Highlighting}
\end{Shaded}

\begin{verbatim}
##      park               biolyr         pack              packcode     
##  Length:864         Min.   :1986   Length:864         Min.   :  2.00  
##  Class :character   1st Qu.:1999   Class :character   1st Qu.: 48.00  
##  Mode  :character   Median :2006   Mode  :character   Median : 86.50  
##                     Mean   :2005                      Mean   : 91.39  
##                     3rd Qu.:2012                      3rd Qu.:133.00  
##                     Max.   :2021                      Max.   :193.00  
##                                                                       
##   packsize_aug       mort_yn          mort_all         mort_lead      
##  Min.   : 0.000   Min.   :0.0000   Min.   : 0.0000   Min.   :0.00000  
##  1st Qu.: 5.000   1st Qu.:0.0000   1st Qu.: 0.0000   1st Qu.:0.00000  
##  Median : 8.000   Median :0.0000   Median : 0.0000   Median :0.00000  
##  Mean   : 8.789   Mean   :0.1956   Mean   : 0.3715   Mean   :0.09552  
##  3rd Qu.:12.000   3rd Qu.:0.0000   3rd Qu.: 0.0000   3rd Qu.:0.00000  
##  Max.   :37.000   Max.   :1.0000   Max.   :24.0000   Max.   :3.00000  
##  NA's   :55                                          NA's   :16       
##   mort_nonlead        reprody1        persisty1     
##  Min.   : 0.0000   Min.   :0.0000   Min.   :0.0000  
##  1st Qu.: 0.0000   1st Qu.:1.0000   1st Qu.:1.0000  
##  Median : 0.0000   Median :1.0000   Median :1.0000  
##  Mean   : 0.2641   Mean   :0.7629   Mean   :0.8865  
##  3rd Qu.: 0.0000   3rd Qu.:1.0000   3rd Qu.:1.0000  
##  Max.   :22.0000   Max.   :1.0000   Max.   :1.0000  
##  NA's   :12        NA's   :71       NA's   :9
\end{verbatim}

\begin{Shaded}
\begin{Highlighting}[]
\FunctionTok{str}\NormalTok{(wolves)}
\end{Highlighting}
\end{Shaded}

\begin{verbatim}
## 'data.frame':    864 obs. of  11 variables:
##  $ park        : chr  "DENA" "DENA" "DENA" "DENA" ...
##  $ biolyr      : int  1996 1991 2017 1996 1992 1994 2007 2007 1995 2003 ...
##  $ pack        : chr  "McKinley River1" "Birch Creek N" "Eagle Gorge" "East Fork" ...
##  $ packcode    : int  89 58 71 72 74 77 101 108 109 53 ...
##  $ packsize_aug: num  12 5 8 13 7 6 10 NA 9 8 ...
##  $ mort_yn     : int  1 1 1 1 1 1 1 1 1 1 ...
##  $ mort_all    : int  4 2 2 2 2 2 2 2 2 1 ...
##  $ mort_lead   : int  2 2 0 0 0 0 1 2 1 1 ...
##  $ mort_nonlead: int  2 0 2 2 2 2 1 0 1 0 ...
##  $ reprody1    : int  0 0 NA 1 NA 0 0 1 0 1 ...
##  $ persisty1   : int  0 0 1 1 1 1 0 1 0 1 ...
\end{verbatim}

Problem 3. (3 points) Which parks/ reserves are represented in the data?
Don't just use the abstract, pull this information from the data.

\begin{Shaded}
\begin{Highlighting}[]
\NormalTok{wolves}\SpecialCharTok{\%\textgreater{}\%}
  \FunctionTok{count}\NormalTok{(park)}
\end{Highlighting}
\end{Shaded}

\begin{verbatim}
##   park   n
## 1 DENA 340
## 2 GNTP  77
## 3  VNP  48
## 4  YNP 248
## 5 YUCH 151
\end{verbatim}

There are five parks, which are DENA(Denali National Park and Preserve),
GNTP(Grand Teton National Park), VNP(Voyageurs National Park),
YNP(Yellowstone National Park) and YUCH(Yukon-Charley Rivers National
Preserve).

Problem 4. (4 points) Which park has the largest number of wolf packs?

DENA

\begin{Shaded}
\begin{Highlighting}[]
\NormalTok{wolves}\SpecialCharTok{\%\textgreater{}\%}
  \FunctionTok{filter}\NormalTok{(park}\SpecialCharTok{==}\StringTok{"DENA"}\NormalTok{)}\SpecialCharTok{\%\textgreater{}\%}
  \FunctionTok{count}\NormalTok{(pack)}\SpecialCharTok{\%\textgreater{}\%}
  \FunctionTok{dim}\NormalTok{()}
\end{Highlighting}
\end{Shaded}

\begin{verbatim}
## [1] 69  2
\end{verbatim}

GNTP

\begin{Shaded}
\begin{Highlighting}[]
\NormalTok{wolves}\SpecialCharTok{\%\textgreater{}\%}
  \FunctionTok{filter}\NormalTok{(park}\SpecialCharTok{==}\StringTok{"GNTP"}\NormalTok{)}\SpecialCharTok{\%\textgreater{}\%}
  \FunctionTok{count}\NormalTok{(pack)}\SpecialCharTok{\%\textgreater{}\%}
  \FunctionTok{dim}\NormalTok{()}
\end{Highlighting}
\end{Shaded}

\begin{verbatim}
## [1] 12  2
\end{verbatim}

VNP

\begin{Shaded}
\begin{Highlighting}[]
\NormalTok{wolves}\SpecialCharTok{\%\textgreater{}\%}
  \FunctionTok{filter}\NormalTok{(park}\SpecialCharTok{==}\StringTok{"VNP"}\NormalTok{)}\SpecialCharTok{\%\textgreater{}\%}
  \FunctionTok{count}\NormalTok{(pack)}\SpecialCharTok{\%\textgreater{}\%}
  \FunctionTok{dim}\NormalTok{()}
\end{Highlighting}
\end{Shaded}

\begin{verbatim}
## [1] 22  2
\end{verbatim}

YNP

\begin{Shaded}
\begin{Highlighting}[]
\NormalTok{wolves}\SpecialCharTok{\%\textgreater{}\%}
  \FunctionTok{filter}\NormalTok{(park}\SpecialCharTok{==}\StringTok{"YNP"}\NormalTok{)}\SpecialCharTok{\%\textgreater{}\%}
  \FunctionTok{count}\NormalTok{(pack)}\SpecialCharTok{\%\textgreater{}\%}
  \FunctionTok{dim}\NormalTok{()}
\end{Highlighting}
\end{Shaded}

\begin{verbatim}
## [1] 46  2
\end{verbatim}

YUCH

\begin{Shaded}
\begin{Highlighting}[]
\NormalTok{wolves}\SpecialCharTok{\%\textgreater{}\%}
  \FunctionTok{filter}\NormalTok{(park}\SpecialCharTok{==}\StringTok{"YUCH"}\NormalTok{)}\SpecialCharTok{\%\textgreater{}\%}
  \FunctionTok{count}\NormalTok{(pack)}\SpecialCharTok{\%\textgreater{}\%}
  \FunctionTok{dim}\NormalTok{()}
\end{Highlighting}
\end{Shaded}

\begin{verbatim}
## [1] 36  2
\end{verbatim}

The DENA have the larges number of wolf packs

Problem 5. (4 points) Which park has the highest total number of
human-caused mortalities \texttt{mort\_all}?

\begin{Shaded}
\begin{Highlighting}[]
\NormalTok{wolves}\SpecialCharTok{\%\textgreater{}\%}
  \FunctionTok{group\_by}\NormalTok{(park)}\SpecialCharTok{\%\textgreater{}\%}
  \FunctionTok{summarize}\NormalTok{(}\AttributeTok{max\_mort\_all=}\FunctionTok{max}\NormalTok{(mort\_all))}
\end{Highlighting}
\end{Shaded}

\begin{verbatim}
## # A tibble: 5 x 2
##   park  max_mort_all
##   <chr>        <int>
## 1 DENA             4
## 2 GNTP             4
## 3 VNP              2
## 4 YNP              4
## 5 YUCH            24
\end{verbatim}

The YUCH have the highest total number of human-caused mortalities

The wolves in
\href{https://www.nps.gov/yell/learn/nature/wolf-restoration.htm}{Yellowstone
National Park} are an incredible conservation success story. Let's focus
our attention on this park.

Problem 6. (2 points) Create a new object ``ynp'' that only includes the
data from Yellowstone National Park.

\begin{Shaded}
\begin{Highlighting}[]
\NormalTok{ynp}\OtherTok{\textless{}{-}}\NormalTok{ wolves}\SpecialCharTok{\%\textgreater{}\%}
    \FunctionTok{filter}\NormalTok{(park}\SpecialCharTok{==}\StringTok{"YNP"}\NormalTok{)}
\end{Highlighting}
\end{Shaded}

Problem 7. (3 points) Among the Yellowstone wolf packs, the
\href{https://www.pbs.org/wnet/nature/in-the-valley-of-the-wolves-the-druid-wolf-pack-story/209/}{Druid
Peak Pack} is one of most famous. What was the average pack size of this
pack for the years represented in the data?

\begin{Shaded}
\begin{Highlighting}[]
\NormalTok{ynp}\SpecialCharTok{\%\textgreater{}\%}
  \FunctionTok{filter}\NormalTok{(pack}\SpecialCharTok{==}\StringTok{"druid"}\NormalTok{)}\SpecialCharTok{\%\textgreater{}\%}
  \FunctionTok{summarize}\NormalTok{(}\AttributeTok{mean\_pack\_size=}\FunctionTok{mean}\NormalTok{(packsize\_aug))}
\end{Highlighting}
\end{Shaded}

\begin{verbatim}
##   mean_pack_size
## 1       13.93333
\end{verbatim}

Problem 8. (4 points) Pack dynamics can be hard to predict- even for
strong packs like the Druid Peak pack. At which year did the Druid Peak
pack have the largest pack size? What do you think happened in 2010?

\begin{Shaded}
\begin{Highlighting}[]
\NormalTok{ynp}\SpecialCharTok{\%\textgreater{}\%}
  \FunctionTok{filter}\NormalTok{(pack}\SpecialCharTok{==}\StringTok{"druid"}\NormalTok{)}\SpecialCharTok{\%\textgreater{}\%}
  \FunctionTok{arrange}\NormalTok{(}\FunctionTok{desc}\NormalTok{(packsize\_aug))}
\end{Highlighting}
\end{Shaded}

\begin{verbatim}
##    park biolyr  pack packcode packsize_aug mort_yn mort_all mort_lead
## 1   YNP   2001 druid       26           37       0        0         0
## 2   YNP   2000 druid       26           27       1        1         0
## 3   YNP   2008 druid       26           21       0        0         0
## 4   YNP   2003 druid       26           18       0        0         0
## 5   YNP   2007 druid       26           18       0        0         0
## 6   YNP   2002 druid       26           16       0        0         0
## 7   YNP   2006 druid       26           15       0        0         0
## 8   YNP   2004 druid       26           13       0        0         0
## 9   YNP   2009 druid       26           12       0        0         0
## 10  YNP   1999 druid       26            9       0        0         0
## 11  YNP   1998 druid       26            8       0        0         0
## 12  YNP   1997 druid       26            5       1        2         1
## 13  YNP   1996 druid       26            5       0        0         0
## 14  YNP   2005 druid       26            5       0        0         0
## 15  YNP   2010 druid       26            0       0        0         0
##    mort_nonlead reprody1 persisty1
## 1             0        1         1
## 2             1        1         1
## 3             0        1         1
## 4             0        1         1
## 5             0        1         1
## 6             0        1         1
## 7             0        1         1
## 8             0        1         1
## 9             0        0         0
## 10            0        1         1
## 11            0        1         1
## 12            1        1         1
## 13            0        1         1
## 14            0        1         1
## 15            0        0        NA
\end{verbatim}

The 2001 is the Druid Peak pack have the largest pack size. On the data
, we can find the pack size become the smallest one. I think he
environment maybe very terrible to let wolves to live. And the legal
challenge results the population of wolves to decress.

Problem 9. (5 points) Among the YNP wolf packs, which one has had the
highest overall persistence \texttt{persisty1} for the years represented
in the data? Look this pack up online and tell me what is unique about
its behavior- specifically, what prey animals does this pack specialize
on?

\begin{Shaded}
\begin{Highlighting}[]
\NormalTok{ynp}\SpecialCharTok{\%\textgreater{}\%}
  \FunctionTok{group\_by}\NormalTok{(pack)}\SpecialCharTok{\%\textgreater{}\%}
  \FunctionTok{summarize}\NormalTok{(}\AttributeTok{total\_persisty=}\FunctionTok{sum}\NormalTok{(persisty1,}\AttributeTok{na.rm =}\NormalTok{ T))}\SpecialCharTok{\%\textgreater{}\%}
  \FunctionTok{arrange}\NormalTok{(}\FunctionTok{desc}\NormalTok{(total\_persisty))}
\end{Highlighting}
\end{Shaded}

\begin{verbatim}
## # A tibble: 46 x 2
##    pack        total_persisty
##    <chr>                <int>
##  1 mollies                 26
##  2 cougar                  20
##  3 yelldelta               18
##  4 druid                   13
##  5 leopold                 12
##  6 agate                   10
##  7 8mile                    9
##  8 canyon                   9
##  9 gibbon/mary              9
## 10 nezperce                 9
## # i 36 more rows
\end{verbatim}

The mollies has had the highest overall persistence \texttt{persisty1}
for the years represented in the data. The Mollie's pack was originally
called the Crystal Creek pack. Wolf kills provide carcasses that are
utilized by a variety of scavengers, including grizzly bears.

Problem 10. (3 points) Perform one analysis or exploration of your
choice on the \texttt{wolves} data. Your answer needs to include at
least two lines of code and not be a summary function.

I'm interesting in which pack in the DENA had the highest verall
persistence \texttt{persisty1} for the years between 1997 to 2007
represented in the data?

\begin{Shaded}
\begin{Highlighting}[]
\NormalTok{wolves}\SpecialCharTok{\%\textgreater{}\%}
  \FunctionTok{filter}\NormalTok{(}\FunctionTok{between}\NormalTok{(biolyr,}\DecValTok{1997}\NormalTok{,}\DecValTok{2007}\NormalTok{),park}\SpecialCharTok{==}\StringTok{"DENA"}\NormalTok{)}\SpecialCharTok{\%\textgreater{}\%}
  \FunctionTok{group\_by}\NormalTok{(pack)}\SpecialCharTok{\%\textgreater{}\%}
  \FunctionTok{summarize}\NormalTok{(}\AttributeTok{total\_persisty=}\FunctionTok{sum}\NormalTok{(persisty1,}\AttributeTok{na.rm =}\NormalTok{ T))}\SpecialCharTok{\%\textgreater{}\%}
  \FunctionTok{arrange}\NormalTok{(}\FunctionTok{desc}\NormalTok{(total\_persisty))}
\end{Highlighting}
\end{Shaded}

\begin{verbatim}
## # A tibble: 38 x 2
##    pack            total_persisty
##    <chr>                    <int>
##  1 East Fork                   11
##  2 McKinley Slough              9
##  3 Kantishna River              8
##  4 Mt Margaret                  8
##  5 100 Mile                     7
##  6 Starr Lake                   7
##  7 Grant Creek                  6
##  8 Pinto Creek                  6
##  9 Straightaway                 5
## 10 Bearpaw                      4
## # i 28 more rows
\end{verbatim}

It is the East Fork.

\end{document}
